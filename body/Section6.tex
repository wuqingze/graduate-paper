\section{总结和展望}
\subsection{研究和工作总结}
本系统是利用闲置的家庭宽带和淘汰的硬盘资源,在基于廉价的通用小型计算机树莓派(英语:Raspberry Pi)
的基础上开发出一个为用户提供文件的备份、共享、存储、访问、搜索、管理等功能,满足多终端访问的在线云盘存储系统。

本文的前期工作是收集市场上的云盘服务的定价、访问速度、系统稳定性和树莓派单片机配置、定价的等数据信息,通过比较价格,安全可
用性等方面,来决定本云盘系统的是否存在开发实现的现实意义。通过数据分析,在和市场上商业的云盘服务比较之后,本系统在特定条件下存在
成本和性能上的优势。

本系统利用LAMP(Linux-Apache-MySQL-PHP)来进行开发系统后端服务器开发,同时提供web客户端和安卓客户端来进行文件的访问和备份。

本系统的web端在主流的浏览器Chrome、Edge、IE11、QQ浏览器、360安全浏览器、360极速浏览器、百度浏览器、opera浏览器、手机端的uc浏
览器、小米系统浏览器等浏览器都可以正常访问。在安卓客户端用户可以上传手机上的图片、照片、文本,可以设置自动备份来自动备份手机中的
文件。

基于树莓派和二手机械硬盘的云盘系统的开发在成本上具有巨大的成本优势,同时在特定网络条件下的文件访问备份效果不错,也是本文的特殊意义所在。
\subsection{展望}
本文实现的云盘系统虽然能够一定程度满足当前项目的需求,但是在易
用性、安全性和定制化方面仍然存在改进和提高。

易用性还需进一步改进。一个非计算机专业的用户要使用本系统,需要掌握在路由器网络端口配置、树莓派系统搭建、
LAMP环境搭建和使用、计算机网络知识、Linux磁盘挂载、一些基本Linux命令和vim文本编辑、Android程序编译等等专业
知识。因此如果需要将本文的云盘系统作为产品推广,需要将一些操纵进行封装,让用户脱离复杂的参数配置而直接一键安装使用。
如果仅仅提供源码和文档,我相信对于大部分用户,甚至是计算机专业的用户都会非常头痛,因此一个完善可直接使用
的软件环境是今后需要努力的方向。

安全性还需进一步改进。因为开发时间仓促,本系统的网络传输协议使用的是HTTP,我们知道HTTP是明文传输,因此很容易
遭遇中间人攻击,传输数据被窃听和数据内容篡改,虽然我系统的使用场景在寝室、家中的局域网内,遭受黑客攻击的
可能性微乎其微,但是本系统有提供外网访问功能,因此还是存在一定的安全风险,所以将系统从HTTP迁移到HTTPS也是
今后努力的方向。

定制化还需进一步改进。本系统提供了两种客户端Web和Android,但是有电脑用户更倾向于使用桌面的客户端
进行数据备份下载,市场上主流的云盘服务百度云就同时用户Web,手机端,桌面端的应用,
人们也有使用桌面软件的使用习惯,因此下一步的改进的方向是开发出桌面客户端和IOS应用,从而覆盖更广的用户。
当然本系统还有一些缺失的功能,比如动态分享、垃圾文件清理、通话记录备份、通信录备份、短信备份等功能还存在缺失,
因此今后功能的添加也是一个努力的方向。
