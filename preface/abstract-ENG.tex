\renewcommand\abstractname{\zihao{-3} Abstract}
\phantomsection
\begin{abstract}
    \addcontentsline{toc}{section}{Abstract}
    \zihao{5}
    \par The widespread use of information technology has brought about the rapid growth of the Internet economy.
     Our society has also entered the era of big data. Everyone who uses the Internet has always faced the problem 
     of data storage access. More than a decade ago, due to the limitations of broadband speed and hardware cost, 
     individual users' needs for data storage, transmission, access, backup, synchronization, and sharing could not
     be fully met. As hardware costs and network tariffs decline, people are increasingly able to store and process 
     massive amounts of data efficiently and securely, while at the same time taking into account the flexibility 
     and speed of cost and deployment adjustments. People can use cloud storage technology to implement personal 
     cloud storage applications with multi-terminal data synchronization, seamless platform connectivity, and resource sharing.
    
    \par This paper first analyzes the market pricing of mainstream cloud disk services in the market, and finds that the 
    pricing of these services is still too expensive for ordinary users. Therefore, I designed a cloud disk system based 
    on Linux MCU Raspberry Pi and developed it. The web client and Android client are designed to provide users with an 
    inexpensive cloud storage service. The web client and the Android client map the server file storage directory to the 
    local directory index, and the user can operate the server file as if the local file is operated, such as the addition, 
    deletion, and modification of the server file. The system adopts the typical MVC architecture design to realize data 
    communication between each logical layer, and provides users with functions such as uploading, downloading, sharing, 
    searching and backuping files.

    \par After completing the requirements analysis and system implementation work, this paper conducts functional tests 
    on the user management, system configuration, and file operations of the system, and performs performance tests on local 
    and server disk read and write and network speed and stability. The test result analysis shows that the cloud disk system 
    has controllable storage resource consumption of the system, and its file read/write throughput rate is about 75 Mb/s, 
    which fully meets the needs of users and has a good user experience.
    
    {\bfseries \zihao{5} Keywords:} {\zihao{5} \KeywordsENG}
\end{abstract}