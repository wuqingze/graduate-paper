\newcommand{\TitleCHS}{基于Linux单片机树莓派的云盘系统的设计与实现} %中文标题

\newcommand{\TitleENG}{Design and Implementation of Cloud Storage System Based on Linux Single Chiptop Raspberry Pi} %英文标题

\newcommand{\Author}{吴清泽} %作者名字

\newcommand{\StudentID}{10152510231} %学号

\newcommand{\Department}{计算机科学与软件工程学院} %学院

\newcommand{\Major}{软件工程} %专业

\newcommand{\Supervisor}{沈佳辰} %导师名字

\newcommand{\AcademicTitle}{工程师} %导师职称

\newcommand{\CompleteYear}{2019} %毕业年份

\newcommand{\CompleteMonth}{6} %毕业月份

\newcommand{\KeywordsCHS}{网盘,云盘,云存储,私人网盘,安卓,树莓派} %中文关键词

\newcommand{\KeywordsENG}{cloud storage, cloud storage, private network storage, Android, Raspberry Pi} %英文关键词

\renewcommand\abstractname{\sffamily\zihao{-3} 摘要}
\phantomsection
\begin{abstract}
	\addcontentsline{toc}{section}{摘要}
	\zihao{5}\rmfamily
	\par 信息科技的广泛应用带来了互联网经济的飞速成长,我们的社会也进入了大数据时代,每个使用互联网的个人始终面临着数据的存储访问的问题。
	十几年前困于宽带速度和硬件成本的限制,个人用户在数据存储、传播、访问、备份、同步、分享等方面的需求都不能得到充分的满足。随着硬件
	成本和网络资费的下降,人们越来越能够高效、安全地存储和处理海量数据,同时兼顾成本和部署调整的灵活性和速度。人们可以利用云存储技术
	实现具备多终端数据同步、平台无缝连接、资源共享等功能的个人云存储应用。

	\par 本文首先分析了市场上主流的云盘服务的市场定价,发现这些服务的定价对于普通用户来说还是太昂贵了,因此,本人设计了一种基于Linux单片机树莓派的云盘系统,
	并开发了web客户端和安卓客户端,旨在为用户提供一种廉价的云存储服务。web客户端和安卓客户端将服务端文件存储目录映射为本地的目录索引,用户可以如同操作本地文
	件一样操作服务端文件,例如服务器文件的增删改查操作。本系统采用典型的MVC体系架构设计实现各逻辑层之间的数据通信,为用户提供文件的上传、下载、分享、搜索、备份等功能。

	\par 在完成需求分析和系统实现的工作之后,本文对系统的用户管理、系统配置、文件操作3个方面进行功能性测试,同时对本地、服务器的磁盘读写和
	网络速度和稳定性做了性能测试。实验结果表明,本云盘系统在对3、4个并发用户的并发场景下文件读写吞吐率可以达到约为75Mb/S,基本上满足系统开始的设计目标,
	同时本系统提供界面美观的客户端,用户体验良好。

	{\bfseries \sffamily\zihao{5} 关键词:} \zihao{5}{\rmfamily \KeywordsCHS}
\end{abstract}